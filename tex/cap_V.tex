 \chapter{Conclusão}
\label{cap:conclu}
% - - - - - - - - - - - - - - - - - - - - - - - - - - - - - - - - - - -

O desenvolvimento deste trabalho acerca da utilização de técnicas de Processamento de Imagens e Visão Computacional para um protótipo de contagem dos palos em um teste palografico proporcionou um aprendizado razoável sobre o tema.  Com conclusão deste estudo podemos responder a questão levantada no Capitulo 1, pode ser verificado que se respeitado algumas restrições conforme Seção 3.3.1 é possível a criação de um protótipo para a contagem dos palos a partir de imagens de celular.

O trabalho resultante é baseado em técnicas de processamento digital de imagens e visão computacional, que são áreas crescentes na computação, devido a sua capacidade de abranger inúmeros segmentos de aplicações. Uma outra particularidade foi a utilização da biblioteca OpenCV, a mesmo demonstrou-se muito oficiente e de simples utilização reverberando no bom resultado final.

Os resultados dos testes mostram um grande potencial do protótipo, pois se mostrou eficaz, devido a sua baixa taxa de erro, o que justifica sua utilização em alternativa à contagem manual.  Além de uma maior precisão na contatem é possível obter o resultado mais rapidamente, o que possibilita um ganha de tempo na rotina do profissional aplicador do teste palográfico.

Como possível trabalhos futuros, pode-se ressaltar:
Ainda que o protótipo tenha se mostrado eficaz, ainda precisa de alguns ajustes para ser realizados a fim de aperfeiçoar os resultados e torná-lo mais robusto à falhas.

Na etapa segmentação poderia ser utilizado outros parâmetros de limiarização afim de melhorar a identificação e separação dos palos. Melhorar por exemplo, a sensibilidade à variação de luminosidade.

Desenvolver o aplicativo mobile para utilizar em tempo real o algoritmo de contagem dos palos, mostrando gráficos da analise quantitativa do teste palografico.

Desenvolver um algoritmo de visão computacional para analise qualitativa do teste palografico utilizando técnicas de inteligencia artificial como Redes Neurais, para classificação dos palos com maior precisão.

A implementação dessas e outras melhorias fazem parte dos trabalhos a serem realizados no futuro.








