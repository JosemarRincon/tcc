\chapter{Descrição da classe \textsf{inf-ufg}}
\label{cap:descr}

%% - - - - - - - - - - - - - - - - - - - - - - - - - - - - - - - - - - -
\section{Opções da classe}
\label{sec:opcoes}
Para usar esta classe num documento \LaTeXe, coloque os arquivos 
\verb|inf-ufg.cls|,\ \verb|inf-ufg.bst|,\ \verb|abnt-num.bst|,\ \verb|atbeginend.sty|\ e \verb|tocloft.sty|\ numa pasta onde o compilador \LaTeX\ pode achá--lo (normalmente na mesma pasta que seu arquivo \verb|.tex|), e defina--o como o estilo do seu documento. Por exemplo, uma dissertação de mestrado que usa o modelo abnt de citações bibliográficas:
\begin{verbatim}
\documentclass[dissertacao,abnt]{inf-ufg}
...
\begin{document}
\end{verbatim}

As opções da classe são \verb|[tese]| (para tese de doutorado), \verb|[dissertacao]| (para dissertação de mestrado), \verb|[monografia]| (para monografia de curso de especialização e \verb|[relatorio]| (para relatório final de curso de graduação). Se nenhuma opção for declarada, o documento é considerado como uma dissertação de mestrado. Se a opção \verb|[abnt]| for utilizada, as citações bibliográficas serão geradas conforme definido pelo grupo de trabalho \textsf{abnt-tex}. Contudo, o mais recomendável é não utilizar essa opção. Com a opção \verb|[nocolorlinks]| todos os {\em links} de navegação no texto ficam na cor preta. O ideal é usar esta opção para gerar o arquivo para impressão, pois a qualidade da impressão dos {\em links} fica superior.


%% - - - - - - - - - - - - - - - - - - - - - - - - - - - - - - - - - - -
\section{Parâmetros da classe}
\label{sec:param}
Os elementos pré-textuais são definidos página por página e dependem da correta definição dos parâmetros listados a seguir (aqueles que contém um texto/valor padrão não precisam ser definidos, caso atenda a situação do autor do texto que está usando a classe \verb|inf-ufg.cls|):

% \setlength{\topsep}{5.2em}
 \begin{itemize}%\addtolength{\itemsep}{-0.7em}
\item \verb|\autor| : Nome completo do autor da tese, começando pelo apelido (ex.: José da Silva);
\item \verb|\autorR| : Nome completo do autor da tese, começando pelo nome (ex.: da Silva, José);
\item \verb|\titulo| : Título da tese, dissertação, monografia ou relatório de conclusão de curso;
\item \verb|\subtitulo| : Se tiver um subtítulo, use este macro para defini--lo;

\item \verb|\cidade| : A cidade de edição. A cidade padrão é \textsf{Goiânia}.
\item \verb|\dia| : Dia do mês da data de defesa (1--31);
\item \verb|\mes| : Mês da data de defesa (1--12);
\item \verb|\ano| : Ano da data de defesa;

\item \verb|\universidade| : Nome completo da universidade. O nome padrão é \textsf{Universidade Federal de Goiás};
\item \verb|\uni| : Sigla da universidade. A sigla padrão é \textsf{UFG};
\item \verb|\unidade| : Nome da unidade acadêmica. O padrão é \textsf{Instituto de Informática};
\item \verb|\departamento| : Nome do departamento, com maiúscula na primeira letra (para o caso de unidades com mais de um departamento);

\item \verb|\programa| : Nome do programa de pós-graduação, com maiúscula na primeira letra. O padrão é \textsf{Computação};
\item \verb|\concentracao| : Nome da área de concentração;

\item \verb|\orientador| : Nome completo do orientador, começando pelo apelido;
\item \verb|\orientadorR| : Nome completo do orientador, começando pelo nome;

\item \verb|\orientadora| : Nome completo da orientadora, começando pelo apelido; use este comando e o próximo se for orientadora e nao orientador.
\item \verb|\orientadoraR| : Nome completo do orientadora, começando pelo nome;

% \item \verb|\CDU| : CDU das publicações do instituto (a perguntar na biblioteca central);
% \item \verb|\paginaspre| : Número de páginas pré-textuais;
% \item \verb|\paginastex| : Número de páginas textuais;
% \item \verb|\altura| : Altura do papel utilizado para a impressão do trabalho. O padrão é o papel A4, de altura 29.7cm;

\item \verb|\coorientador| : Nome completo do co--orientador, começando pelo apelido;
\item \verb|\coorientadorR| : Nome completo do co--orientador, começando pelo nome;

\item \verb|\coorientadora| : Nome completo da coorientadora, começando pelo apelido; use este comando e o próximo se for coorientadora e nao coorientador.
\item \verb|\coorientadoraR| : Nome completo do coorientadora, começando pelo nome;

\item \verb|\universidadeco| : Nome da universidade do coorientador;
\item \verb|\unico| : Sigla da universidade do coorientador;
\item \verb|\unidadeco| : Nome da unidade acadêmica do coorientador.\footnote{Se não tiver um co--orientador, não defina esses últimos sete parâmetros.}
\end{itemize}

%% - - - - - - - - - - - - - - - - - - - - - - - - - - - - - - - - - - -
\section{Elementos Pré--Textuais}
\label{sec:pre}
Os elementos pré--textuais são definidos página por página, conforme descritos a seguir:

\paragraph{capa\\}
\verb|\capa| : Gera o modelo da capa externa do trabalho. Esta página servirá apenas como modelo para a encadernação da versão final do texto. Nenhum dado é necessário.

\paragraph{publicação\\}
\verb|\publica| : Gera a autorização para publicação do trabalho em formato eletrônico e disponibilização do mesmo na biblioteca virtual da UFG.

\paragraph{rosto\\}
\verb|\rosto| : Gera a folha de rosto, a qual é a primeira folha interna do trabalho. Nenhum dado é necessário.

\paragraph{aprovação\\}
\verb|\aprovacao| : ambiente para a reprodução do termo de aprovação da Banca Examinadora da tese ou dissertação.

\paragraph{banca\\}
\verb|\banca| : Entrada para o nome dos examinadores, exceto o(s) orientador(es).

\noindent\verb|\profa| : Entrada para o nome das examinadoras, exceto o(s) orientador(es).

\paragraph{direitos\\}
\verb|\direitos| : Macro com 2 argumentos para gerar os direitos autorais, o perfil do aluno e a ficha catalográfica da Biblioteca Central da UFG.
\begin{itemize}
\item O primeiro argumento é o Perfil do aluno; e
\item O segundo argumento é a lista das palavras--chaves para a Ficha Catalográfica.
\end{itemize}

\paragraph{dedicatória\\}
\verb|\dedicatoria| : ambiente para escrever a dedicatória. É possível trocar o espaçamento dentro desse ambiente do mesmo jeito que no \LaTeX\ padrão.

\paragraph{agradecimentos\\}
\verb|\agradecimentos| : ambiente para escrever os agradecimentos. É possível trocar o espaçamento dentro desse ambiente do mesmo jeito que no \LaTeX\ padrão.

\paragraph{resumo\\}
\verb|\chaves| : A lista das palavras chaves, separadas por `;'. Deve ser definido antes do ambiente \verb|\resumo|, o qual é usado para escrever o resumo em português.

\paragraph{abstract\\}
\verb|\keys| : A lista das palavras chaves em inglês, separadas por `;'. Deve ser definido antes do ambiente \verb|\abstract|, o qual contém 1 argumento e é usado escrever o resumo em inglês. O argumento deve ser o título do trabalho em inglês.

\paragraph{tabelas\\}
\verb|\tabelas| : Macro com 1 argumento opcional para gerar as tabelas. O argumento pode ser:
\begin{itemize}
 \item nada [] : gera apenas o sumário;
 \item \textsf{fig} : gera o sumário e uma lista de figuras;
 \item \textsf{tab} : gera o sumário e uma lista de tabelas;
 \item \textsf{alg} : gera o sumário e uma lista de algoritmos;
 \item \textsf{cod} : gera o sumário e uma lista de programas.
% \item \textsf{figtab} : gera o sumário, uma lista de tabelas, e uma lista de figuras;
% \item \textsf{figtabalg} : gera o sumário e listas de tabelas, de figuras e de algoritmos;
% \item \textsf{figtabalgcod} : gera o sumário e listas de tabelas, de figuras, de algoritmos e de programas;
% \item (qualquer outra coisa) : gera somente o sumário.
\end{itemize}

Pode-se usar qualquer combinação dessas opções. Por exemplo:
\begin{itemize}
 \item \textsf{figtab} : gera o sumário e listas de figuras e tabelas,
 \item \textsf{figtabcod} : gera o sumário e listas de figuras, tabelas e códigos de programas;
 \item \textsf{figtabalg} : gera o sumário e listas de figuras, tabelas e algoritmos;
 \item \textsf{figtabalgcod} : gera o sumário e listas de figuras, tabelas, algoritmos e códigos de programas
\end{itemize}

\paragraph{epígrafe\\}
\verb|\epigrafe| : Macro com 3 argumentos que permite editar um epígrafe. O primeiro argumento é o texto da citação. O segundo argumento é o nome do autor da citação. O terceiro argumento é o título da referência à qual a citação pertence.
